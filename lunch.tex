\chapter{Lunch}

\section{Rice}
\subsubsection{Ingredients}
\begin{itemize}
    \item white rice:
    \begin{enumerate}
        \item White rice
        \item Cooking oil
    \end{enumerate}
    \item yellow rice:\\ In addition to the ingredients from the white rice\ldots
    \begin{enumerate}
        \item Turmeric
        \item Cinnamon
        \item Black pepper
    \end{enumerate}
\end{itemize}

\subsubsection{Cooking}
\begin{enumerate}
    \item Add a one to one (rice:water) ratio of water to pot.
    \item Wash the rice until the water is somewhat clear. You're going to have to add and remove water a few times to achieve a non starchy and clean rice.
    \item If the rice grain is thick. Let the rice rest in water for at least 10 minutes.
    \item Season the pot water with a spoon of cooking oil and salt to taste. The quantity of oil can be changed depending on the quantity of rice.\\
        - \textit{For yellow rice Season water with tea spoon of turmeric, cinnamon and a pinch of black pepper. All can be adjusted by eye and taste.}
    \item Add rice only after the water has come to a boil and cover with a lid.
    \item Turn down the heat and stir the rice occasionally to avoid burning the rice once most of the water has boiled. Try to preserve the steam by avoiding open the lid too many times as that is what cooks the rice.
    \item Finally remove the heat once the rice is cooked through and through. Taste test to ensure it is cooked well before serving!
\end{enumerate}


\section{Potatoes and Peas sauce}
\subsubsection{Ingredients}
\begin{enumerate}
    \item Turmeric
    \item Black pepper
    \item Tomato sauce
    \item Pomegranate sauce
    \item Chicken (of any kind)
    \item Carrot
    \item Frozen Green Peas
    \item Potatoes
\end{enumerate}

\subsubsection{Prep}
Prepare the following:
\begin{enumerate}
    \item A large potato (or two medium) that has been cut into cubes.
    \item A cup and a half of frozen green peas.
    \item 2 carrots that have been cut into cubes.
    \item Dice half an onion.
    \item and mince 1 cloves of garlic.
\end{enumerate}

\subsubsection{Cooking}
\begin{enumerate}
    \item Add the onion, garlic and the chicken into the pot.
    \item Add $\frac{1}{2}$ spoon of black pepper powder.
    \item Add 2 spoons of turmeric.
    \item Add pomegranate sauce to taste.
    \item Add 4 spoons of cooking oil and turn the heat onto medium.
    \item Once the chicken starts to change color Add the vegetables to the pot.
    \item Once the vegetables have absorbed the spices, add 5 spoons of tomato sauce.
    \item Then add water to get the right texture.
    \item Add about $\frac{1}{2}$ salt to taste.
    \item Let it cook for 45 to 60 minute.
\end{enumerate}

\section{Pomegranate Chicken}
\subsubsection{Ingredients}
\begin{enumerate}
    \item 1 Onion.
    \item 2 Garlic cloves.
    \item \(\frac{1}{2}\) a lemon.
    \item Pomegranate sauce
    \item 3 or 4 chicken legs
\end{enumerate}

\subsubsection{Cooking}
\begin{enumerate}
    \item Cut an onion and 2 garlic cloves into slides and add them into a pot.
    \item Add 4 spoons of oil.
    \item Add the chicken into the pot and turn on the heat.
    \item Once the chicken start to change color.
    \item Add about half a tea cup of pomegranate sauce to taste.
    \item Add salt and pepper to taste.
    \item Squeeze half a lemon into all the contents of the pot. Cut slices of lemon peel to liking.
    \item Add about \(1\frac{1}{2}\) cup water.
    \item Let the contents cook for an hour to an hour and a half. Don't let the pot dry out, always make sure there is water in the pot boiling.
\end{enumerate}

\section{Ribeye Stake}
\subsubsection{Ingredients}
\begin{enumerate}
    \item Ribeye meat
    \item $\frac{1}{4}$ a stick of butter
    \item 1 garlic clove
    \item A bit of thyme
    \item Some olive oil
\end{enumerate}

\subsubsection{Seasoning the Meat}
Rub a healthy amount of salt and pepper into the ribeye and that is basically it.

\subsubsection{Cooking}
\begin{enumerate}
    \item Add a small amount of olive oil to the pan, just enough to stop the stake from sticking. Make sure the pan is piping hot. \textit{we don't want to boil the stake we want to sear it.}
    \item Place stake going away from you.
    \item Flip stake after 30 seconds and let it sear for another 30 seconds.
    \item Turn heat down and add the butter along with the thyme.
    \item Flip every minute for even cooking while basting the stake with the brown golden butter/olive oil.
\end{enumerate}

\section{Mashed potatoes}
\subsubsection{Ingredients}
\begin{enumerate}
    \item Potatoes
    \item Butter
    \item Milk
    \item Salt
\end{enumerate}

\subsubsection{Cooking}
\begin{enumerate}
    \item Peal the potatoes.
    \item Cut into even quarters.
    \item Add the potatoes to a pot.
    \item Add enough water to cover the potatoes by 2 to 3cm.
    \item Cover and let the potatoes boil.
    \item Once the potatoes are soft enough to be stabbed though using a fork, pour the contents of the pot into a colander.
    \item Add butter, salt and a dash of milk to taste.
    \item Combine everything together by mashing the potatoes.
\end{enumerate}

\section{Pizza}
\subsubsection{Ingredients}
\begin{enumerate}
    \item $2 \frac{1}{4}$ cups warm water
    \item $1$ tbsp sugar
    \item $1$ tbsp active dry yeast
    \item $2$ tbsp olive oil
    \item $1$ tbsp salt
    \item $5$ cups flour
\end{enumerate}

Half the amount would be:
\begin{enumerate}
    \item $1 \frac{1}{8}$ cups warm water
    \item $\frac{1}{2}$ tbsp sugar
    \item $\frac{1}{2}$ tbsp active dry yeast
    \item $1$ tbsp olive oil
    \item $\frac{1}{2}$ tbsp salt
    \item $2 \frac{1}{2}$ cups flour
\end{enumerate}

\subsubsection{Making the dough}
\begin{enumerate}
    \item Start the dough by combining the water, sugar and yeast in a large bowl and let sit for a few minutes. If the yeast goes foamy, it's alive and you're good
    \item Add the olive oil and salt and the flour.
    \item Add just enough additional flour to keep the dough workable (i.e. not too sticky) and kneed until you can stretch some of the dough into a thin sheet without it tearing. (NOTE: You will probably need to add a lot more flour. The quantity I give here is just a base line to get your started.)
    \item Let it rise for 2 hours then divide equally and store them in the fridge (optimally for 7 days) or freezer.
\end{enumerate}

\subsubsection{Baking}
\begin{enumerate}
    \item Preheat the oven at the highest temperature (ideally convection).
    \item Take the cold dough out of the fridge and dust it in flour. Stretch to the size wanted.
    \item Add the tomato sauce and the toppings needed.
    \item until the crust is well-browned and the cheese has browned a bit (but, ideally, has not started oozing out an orange grease layer), 6-7 minutes.
\end{enumerate}

\section{Meat Teshreeb}
\subsubsection{Ingredients}
\begin{itemize}
    \item $1$ Onion, cut into large chunks
    \item $4$ tbsp of Tomato paste
    \item $4$ tbsp Cooking Oil
    \item Iraqi bread
    \item Meat
    \item $\frac{1}{2}$ a can of chickpeas, to thicken the sauce
    \item Loomybsra, black spheres, need to find the English name\ldots
\end{itemize}

\subsubsection{Cooking}
\begin{enumerate}
    \item Cut Onion, into $4$ large chunks.
    \item Add $4$ tbsp of oil to the pot, along with the onions and meat.
    \item Add water until meat is covered.
    \item filter water from the chickpeas can and add the chickpeas to the pot.
    \item Add in $2$ pieces of Loomybsra.
    \item Let is boil on a medium heat for an hour.
\end{enumerate}
