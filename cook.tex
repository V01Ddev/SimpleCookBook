
\chapter{Breakfasts}

\chapter{Lunch}
\section{White Rice}
\subsection{Ingredients}
\begin{enumerate}
    \item White rice
    \item Cooking oil
\end{enumerate}
\subsection{Cooking}
\begin{enumerate}
    \item Add a one to one (rice:water) ratio of water to pot.
    \item Season the water with a spoon of cooking oil and salt to taste. The quantity of oil can be changed depending on the quantity of rice.
    \item As we wait for the water to start to boil, keep wash the rice until the water is somewhat clear. You're going to have to add and remove water a few times to achieve a non starchy and clean rice.
    \item Add rice only after the water has come to a boil and cover with a lid.
    \item Turn down the heat and stir the rice occasionally to avoid burning the rice once most of the water has boiled. Try to preserve the steam by avoiding open the lid too many times as that is what cooks the rice.
    \item Finally remove the heat once the rice is cooked through and through. Taste test to ensure it is cooked well before serving!
\end{enumerate}

\section{Ribeye Stake}
\subsection{Ingredients}
\begin{enumerate}
    \item Ribeye meat
    \item $\frac{1}{4}$ a stick of butter
    \item 1 garlic clove
    \item A bit of thyme
    \item Some olive oil
\end{enumerate}

\subsection{Seasoning the Meat}
Rub a healthy amount of salt and pepper into the ribeye and that is basically it.

\subsection{Cooking}
\begin{enumerate}
    \item Add a small amount of olive oil to the pan, just enough to stop the stake from sticking. Make sure the pan is piping hot. \textit{we don't want to boil the stake we want to sear it.}
    \item Place stake going away from you.
    \item Flip stake after 30 seconds and let it sear for another 30 seconds.
    \item Turn heat down and add the butter along with the thyme.
    \item Flip every minute for even cooking while basting the stake with the brown golden butter/olive oil.
\end{enumerate}


\chapter{Desserts}

